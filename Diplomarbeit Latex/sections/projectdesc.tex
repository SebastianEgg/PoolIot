



\section{Projektmanagement}
Das Projekt ist im Softwarebereich mittels Scrum und im Hardwarebereich mittels einer hybriden Projektentwicklung entwickelt worden.

\subsection{Scrum}
Scrum ist ein agiles Vorgehensmodell für Projektmanagement
und kommt zum Einsatz wenn viele Entwicklungsprojekte zu komplex sind, um sie in einen vollumfassenden Plan zu fassen.
Aus diesem Grund ist ein wesentlicher Teil der Anforderungen und Lösungen zu Beginn unklar. 
Diese werden schrittweise erarbeitet, indem Zwischenergebnisse geschaffen werden.
Durch Scrum wird die Projektlaufzeit in Etappen, so genannte Sprints, eingeteilt und
am Ende jedes Sprints soll jeweils ein funktionsfähiges Zwischenprodukt stehen.
Die Einteilung wie lange ein Sprint dauert wird intern geregelt.
In diesem Projekt war die Dauer eines Sprintes einen Monat lang.

\subsection{Hybride Projektentwicklung}
Hybride Projektentwicklung wird meist im Hardwarebereich verwendet.
Es ermöglicht eine schnellere Produktion von Hardwareteilen.
In diesem Projekt ist es bei dem Zusammenbau für den Rasbperray verwendet
und für die Produktion einer Schwimmboje vorgesehen worden.



\subsection{Zusätzliche Vereinbarungen der Diplomarbeit}
Zusätlich zu den Projektentwicklungskonzepten wurde jede zweite Woche ein Projektbericht veröffentlicht, um die Diplomarbeitsbetreuer auf dem Laufenden zu halten.

\begin{figure}[H]
    \centering
    \includegraphics[width=1\textwidth]{pics/AusschnittProjektberichte.png}
    \caption{Ausschnitt aus den Projektberichten}
\end{figure}


In den Projektberichten sind die fertigen Meilensteine und die  Meilensteine an denen gearbeitet wurde, mittels einer Projektampel veranschaulicht.
Eine Projektampel sagt aus, ob sich ein Projekt in Gefahr, in Verzug oder im grünen Bereich befindet.


\begin{figure}[H]
    \centering
    \includegraphics[width=0.7\textwidth]{pics/Projektampel.png}
    \caption{Visualisierung einer Projektampel}
\end{figure}

\section{Backend Projekt-Überblick}
\setauthor{Sebastian Egger}

Das Backend setzt sich aus 9 Projekten zusammen, welches mithilfe des von Herrn DI Professor Gerhard Gehrer zur Verfügung gestellten Frameworkes SmartNQuick entwickelt wurde.
Das Backend wurde in der Sprache C\# in .Net 5 programmiert und als Code Editor wurde
Visual Studio 2019 und 2022 verwendet.
 


\begin{figure}[H]
    \centering
    \includegraphics[width=0.7\textwidth]{pics/Projektmappe.JPG}
    \caption{Projektmappe}
\end{figure}


\begin{itemize}
    \item CommonBase
    \\
    In CommonBase befinden sich Klassen und Methoden, die wiederverwendbar sind, um Codeverdoppelung zu vermeiden.
\end{itemize}


\begin{itemize}
    \item SnQPoolIot.Adapters
    \\
    SnQPoolIot.Adapters bietet einen direkten Zugriff auf die Logic.
Der Zugriff auf die Logic kann dadurch entweder direkt erfolgen oder per Rest über die WebApi.
\end{itemize}

\begin{itemize}
    \item SnQPoolIot.WebApi
    \\
    Der Zugriff auf die Messwerte wird durch Rest-Zugriffe in SnQPoolIot.WebApi provided.
    Auf die Daten kann aber nur per Login mit einem gültigen Account zugegriffen werden.
    Genaueres zu den einzelnen HTTP-Requests ist im Kapitel HTTP und Verwendung in unserem Backend zu finden.
\end{itemize}

\begin{itemize}
    \item SnQPoolIot.Contracts
    \\
    SnQPoolIot.Contracts beinhaltet alle notwendigen Schnittstellen und Enumerationen des Projektes.
    Hier werden die Entitäten als Interfaces angelegt.
\end{itemize}


\begin{itemize}
    \item SnQPoolIot.Logic
    \\
    SnQPoolIot.Logic ist das Kernstück des Projektes. 
    Durch die Logic können alle Daten aus der Datenbank verwendet werden. 
    Die Logic verbindet sich mit einer Sqlite Datenbank. Der Zugriff und das Erzeugen der Datenbank wird mittels Entityframework.Sqlite durchgeführt.
\end{itemize}

\begin{itemize}
    \item SnQPoolIot.Transfer
    SnQPoolIot.Transfer verwaltet die Transferobjekte für den Datenaustausch zwischen den Layern.
\end{itemize}


\begin{itemize}
    \item SnQPoolIot.AspMvc
    SnQPoolIot.AspMvc ist ein Ersatz für das Frontend.
    Hier werden die Funktionen z.B.: das Einloggen eines Users oder Anzeigen von Messwerten dargestellt.
\end{itemize}

\begin{itemize}
    \item SnQPoolIot.ConApp
    In SnQPoolIot.ConApp werden User mit verschiedenen Rechten angelegt, die für die Authentifizierung benötigt werden.
\end{itemize}



\subsection{SnQPoolIot.Logic}
Wie  bereits im Backend Projekt-Überblick beschrieben, befindet sich in SnQPoolIot.Logic die Datenbank mit den Zugriffen.
Die Datenbank wird mithilfe des NuggetPackage Microsoft.Entityframework.Sqlite, den Befehlen:
dotnet ef migrations add InitDb und dotnet ef database update, welche in der Developer-PowerShell 
im Visual Studio ausgeführt werden müssen um Migrations zu erzugen und um die Datenbank mit den erzeugten Migrations upzudaten, 
und einem DBContext, welcher die Configuration der Datenbank mit sich bringt, automatisch erstellt. 

\begin{figure}[H]
    \centering
    \includegraphics[width=1\textwidth]{pics/EntityFrameworkSqlLiteNuggetPackage.JPG}
    \caption{NuggetPackage für Entityframework mit Sqlite}
\end{figure}


\begin{figure}[H]
    \centering
    \includegraphics[width=1\textwidth]{pics/DeveloperPowerShellMigration.JPG}
    \caption{Befehl zum Erzeugen von Migationen}
\end{figure}

\begin{figure}[H]
    \centering
    \includegraphics[width=1\textwidth]{pics/MigrationCreated.png}
    \caption{Erzeugte Migrationen in der Logik}
\end{figure}

\begin{figure}[H]
    \centering
    \includegraphics[width=1\textwidth]{pics/DBContext.JPG}
    \caption{DBContext zum Erzeugen einer Datenbank}
\end{figure}

Nun zu den Datenbank Zugriffen. Um auf die Messwerte zugreifen zu können, muss sich ein User zuerst authentifizieren.
Damit sich eine User authentifizieren kann benötigt er einen Account mit E-Mail und Passwort, welche in der Datenbank gespeichert sind.
Für die Authentifizierung werden die nachstehenden Tabellen benötigt:

\begin{itemize}
    \item Identity\\
    In dieser Tabelle befinden sich alle User-Accounts mit E-Mail gehastem Passwort.

    \begin{figure}[H]
        \centering
        \includegraphics[width=0.7\textwidth]{pics/IdentityTableStructur.JPG}
        \caption{Identity Tabelle mit den dazugehörigen Tabellenspalten}
    \end{figure}

\end{itemize}


\begin{itemize}
    \item Role\\
    In dieser Tabelle sind alle Rollen, die es gibt, vorzufinden.

    \begin{figure}[H]
        \centering
        \includegraphics[width=1\textwidth]{pics/RoleTableStructure.JPG}
        \caption{Role Tabelle mit den dazugehörigen Tabellenspalten}
    \end{figure}

\end{itemize}

\begin{itemize}
    \item IdentityXRole\\
    Diese Tabelle weist einem Benutzer eine Rolle zu.
    \begin{figure}[H]
        \centering
        \includegraphics[width=1\textwidth]{pics/IdentityXRoleStructure.JPG}
        \caption{IdentityXRole Tabelle mit den dazugehörigen Tabellenspalten}
    \end{figure}
\end{itemize}


\begin{itemize}
    \item LogginSession\\
    Diese Tabelle zeigt alle Logins mit dem dazugehörigen User und dem SessionToken, mit dem sich der User eingeloggt hat, an.
    \begin{figure}[H]
        \centering
        \includegraphics[width=1\textwidth]{pics/LoginSessionStructure.JPG}
        \caption{LogginSession Tabelle mit den dazugehörigen Tabellenspalten}
    \end{figure}
\end{itemize}

Sobald sich ein User authentifiziert hat, wurde zugleich eine neue Session erstellt. Nur bei ausreichender Berechtigung des authentifizierten Benutzers kann dieser sich nun die Messwerte ansehen.
Die nachstehenden Tabellen dienen zum Erfassen der Messwerte:

\begin{itemize}
    \item Sensor\\
    Diese Tabelle beinhaltet den Namen eines Sensors.
    \begin{figure}[H]
        \centering
        \includegraphics[width=1\textwidth]{pics/SensorTableStructure.JPG}
        \caption{Sensor Tabelle mit den dazugehörigen Tabellenspalten}
    \end{figure}
\end{itemize}

\begin{itemize}
    \item SensorData\\
    Diese Tabelle beinhaltet die Messwerte aller Sensoren und jeder Messwert ist einem Sensor zugeordnet.
    \begin{figure}[H]
        \centering
        \includegraphics[width=1\textwidth]{pics/SensorDataTableStructure.JPG}
        \caption{SensorData Tabelle mit den dazugehörigen Tabellenspalten}
    \end{figure}
\end{itemize}


\subsubsection{Logging in unserem Projekt}
Um Fehlermeldungen oder Warnungen im Projekt zu speichern, wurde eine Klasse mit dem Namen LogWriter programmiert.
Diese Meldeungen werden in einer .txt gespeichert und können jederzeit gelesen werden.
\\Im nachstehenden Code ist die LogWriter Klasse zu sehen:
\\
\begin{lstlisting}
using System;
using System.Collections.Generic;
using System.IO;
using System.Linq;
using System.Reflection;
using System.Text;
using System.Threading.Tasks;

namespace SnQPoolIot.Logic.Entities.Business.Logging
{
    public class LogWriter
    {
        private string m_exePath = string.Empty;
        private static LogWriter _instance = null;

        public static LogWriter Instance
        {
            get
            {
                if(_instance == null)
                {
                    _instance = new LogWriter();
                }
                return _instance;
            }
        }

        private LogWriter()
        {

        }
        public void LogWrite(string logMessage)
        {
            m_exePath = Path.GetDirectoryName(Assembly.GetExecutingAssembly().Location);
            try
            {
                using StreamWriter w = File.AppendText(m_exePath + "\\" + "log.txt");
                Log(logMessage, w);
            }
            catch (Exception)
            {
            }
        }
        public static void Log(string logMessage, TextWriter txtWriter)
        {
            try
            {
                txtWriter.Write("\r\nLog Entry : ");
                txtWriter.WriteLine("{0} {1}", DateTime.Now.ToLongTimeString(),
                    DateTime.Now.ToLongDateString());
                txtWriter.WriteLine("  :{0}", logMessage);
                txtWriter.WriteLine("-------------------------------");
            }
            catch (Exception )
            {
            }
        }
    }
}
\end{lstlisting}

Der Methode LogWrite wird durch einen Übergabeparameter die auszugebende Fehlermeldungen übergeben. Nach Ermittlung der Datei wird durch Aufruf der Methode Log 
die Fehlermeldung inklusive Zeitstempel in der Datei gespeichert.
\\
\\

\subsection{SnQPoolIot.WebApi}

\subsubsection*{HTTP und Verwendung im Backend}

HTTP ausgeschrieben Hypertext Transfer Protocoll wird zum Laden von Webseiten im Projekt verwendet.
Die verwendeten HTTP-Requests und das dazugehörige Routing ist im Projekt SnQPoolIot.WebApi implementiert.

Im Folgenden ist ein Ausschnitt des Aufbaus der WebApi dargestellt:

\begin{figure}[H]
    \centering
    \includegraphics[width=1\textwidth]{pics/HTTPRequestsWebApi.JPG}
    \caption{Aufbau der WebApi}
\end{figure}


Das Routing der Websiten wird über die Controller gehandelt. 
Der meist genützte Controller in diesem Projekt ist der GenericController, denn er bietet allen abgeleiteten Controllern seine bereitgestellten Funktionen zur Verwendung an.

\begin{figure}[H]
    \centering
    \includegraphics[width=1\textwidth]{pics/GenericControllerWebApi.JPG}
    \caption{Auszug GenericController der WebApi}
\end{figure}

Wie am Beispiel des Codeauszuges zu sehen, beinhaltet der Generic Controller einige Methoden.
\\Für die Ermittlung der Daten wird zum Beispiel eine Get-Methode \""GetAllAsync\""  eingesetzt.
\\Die Codezeile [HttpGet(\""/api/[controller]\"")] drückt aus, dass es sich bei dieser Methode um eine Get-Methode handelt 
und veranschaulicht auch das Routing der WebApi.
\\All jene Methoden, die im GenericController definiert sind, stehen demnach allen von ihm abgeleiteten Controllern zur Verfügung und können direkt verwendet werden.

Im nachstehenden Beispiel wird eine Ableitung vom GenericController dargestellt.

\begin{figure}[H]
    \centering
    \includegraphics[width=0.7\textwidth]{pics/VerwendungGenereicController.JPG}
    \caption{Anwendung des GenericControllers der WebApi}
\end{figure}


\subsubsection*{Verwendung von Swagger im Projekt}
Zum Allgemeinen Überblick aller HTTP-Requests wird Swagger verwendet.

\begin{figure}[H]
    \centering
    \includegraphics[width=0.8\textwidth]{pics/SwaggerImplementation.JPG}
    \caption{Implementierung von Swagger in der WebApi}
\end{figure}

In den nachstehenden Zeilen findet sich eine kurze Ablaufbeschreibung des Codes:
Sobald die Methode ConfigureServices aufgerufen wird, wird dem service ein neuer Controller angelegt und die benötigte Konfiguration des Swaggers mit übergeben. 

Das options.SwaggerDoc gibt eine kurze Beschreibung über die WebApi an.
Im darauf folgenden Schritt wird die Authentifizierung im Projekt mittels Swagger durchgeführt. Dadurch wird gewährleistet, dass nur eingeloggte Benutzer die Möglichkeit haben Zugriffe auf HTTP-Requests durchzuführen.
Sobald die Methode fertig ausgeführt wurde startet im Browser eine Website mit allen HTTP-Requests die im Projekt implementiert sind.

\subsubsection*{Unterstützte HTTP-Requests des Projektes}
Mit Hilfe von Swagger wurden die in den nachstehenden Grafiken ersichtlichen HTTP-Requests des Backends dokumentiert:

\begin{figure}[H]
    \flushleft
    \includegraphics[width=1.6\textwidth]{pics/WebApiRequests1.JPG}
    \caption{HTTP-Requests des Projektes}
\end{figure}

\begin{figure}[H]
    \centering
    \includegraphics[width=1.6\textwidth]{pics/WebApiRequests2.JPG}
    \caption{HTTP-Requests des Projektes}
\end{figure}

\begin{figure}[H]
    \centering
    \includegraphics[width=1.6\textwidth]{pics/WebApiRequests3.JPG}
    \caption{HTTP-Requests des Projektes}
\end{figure}

\begin{figure}[H]
    \centering
    \includegraphics[width=1.6\textwidth]{pics/WebApiRequests4.JPG}
    \caption{HTTP-Requests des Projektes}
\end{figure}

\begin{figure}[H]
    \centering
    \includegraphics[width=1.6\textwidth]{pics/WebApiRequests5.JPG}
    \caption{HTTP-Requests des Projektes}
\end{figure}

\begin{figure}[H]
    \centering
    \includegraphics[width=1.6\textwidth]{pics/WebApiRequests6.JPG}
    \caption{HTTP-Requests des Projektes}
\end{figure}

\begin{figure}[H]
    \centering
    \includegraphics[width=1.6\textwidth]{pics/WebApiRequests7.JPG}
    \caption{HTTP-Requests des Projektes}
\end{figure}

\begin{figure}[H]
    \centering
    \includegraphics[width=1.6\textwidth]{pics/WebApiRequests8.JPG}
    \caption{HTTP-Requests des Projektes}
\end{figure}

\begin{figure}[H]
    \centering
    \includegraphics[width=1.6\textwidth]{pics/WebApiRequests9.JPG}
    \caption{HTTP-Requests des Projektes}
\end{figure}

\subsubsection{Implementierung von MQTT in unserem Projekt}
Das Einlesen und Überprüfen von den Messwerten soll stattfinden, sobald die WebApi läuft.
Aus diesem Grund wird die RuleEngine, 
welche den Aufruf für das Einlesen und Überprüfen der Messwerte beinhaltet, instanziert, welche gleichzeitig wie die WebApi startet.

\begin{figure}[H]
    \centering
    \includegraphics[width=0.6\textwidth]{pics/RuleEngineCall.png}
    \caption{Aufruf der RuleEngine}
\end{figure}

\begin{figure}[H]
    \centering
    \includegraphics[width=0.9\textwidth]{pics/InstanzierenDerRuleEngineUndAufrufenDerMQTTMesswerte.png}
    \caption{Instanzieren der RuleEngine und der Sensoren zum Einlesen der Messwerte}
\end{figure}

Die einelnen Sensoren, welche in der obigen Abbildung in die Datenbank gespeichert werden,
befinden sich in dem nachstehenden enum:

\begin{figure}[H]
    \centering
    \includegraphics[width=0.9\textwidth]{pics/SensorEnum.JPG}
    \caption{Liste der Sensoren gespeichert in einem Enum}
\end{figure}

Auf diese Sensoren wird nun die Methode StartMqttClientAndRegisterObserverAsync angewendet,
welche in den nachstehenden Abbildungen und Texten beschrieben wird:

Damit das Backend einen Wert von einem Topic bekommen kann, muss auf dieses Topic scubscribed werden.
In C\# gibt es eigene NuggetPackages zum Subscriben. \\
Nachdem die NuggetPackages installiert worden sind,
kann nun ein MQTT Client mittels einer MQTTFactory erzeugt werden. Damit der MQTT Client auf den richtigen
Broker subscriben kann, muss er sogenannte MQTTClientOpions, mittels eines MQTTClientOpionsBuilder, auf sich zugewiesen bekommen.
\\Zu diesen MQTTClientOpions zählt die ClientId, die Adresse, der Username und das Passwort des Brokers, sowie
der Port auf dem der Broker läuft.
\begin{figure}[H]
    \centering
    \includegraphics[width=1.3\textwidth]{pics/StartMqttClientConnections.JPG}  
    \caption{MQTT Client Konfiguration}
\end{figure}
Sobald die Konfiguration eingetragen wurde, wird dem MQTT Client nun mitgeteilt, was zu tun ist, 
wenn sich der MQTT Client connected, disconnected oder eine Nachricht bekommt.

\begin{figure}[H]
    \centering
    \includegraphics[width=1.3\textwidth]{pics/StartMqttClientCallMethods.JPG}  
    \caption{MQTT Cient HandlerDelegate, Subscriben und Starten}
\end{figure}

Wenn sich ein MQTT Client connected oder disconnected, wird dies 
,durch die selbstgeschrieben Logging Klasse, mitgeloggt.

\begin{figure}[H]
    \centering
    \includegraphics[width=1.1\textwidth]{pics/StartMqttClientConnectAndDisconnect.JPG}  
    \caption{MQTT Cient HandlerDelegate, Subscriben und Starten}
\end{figure}

Auch beim Empfangen einer neuen Message wird dies durch die Logging Klasse mitgeloggt,
 jedoch ist dies nicht die Hauptfunktion der Methode MqttOneNewMessage, sondern 
 das Speichern (Inserten) von Messwerten in die Datenbank.

 \begin{figure}[H]
    \centering
    \includegraphics[width=1.1\textwidth]{pics/StartMqttClientMqttOnNewMessageAsync.JPG}  
    \caption{MQTT Cient beim Empfangen eines neuen Messwertes}
\end{figure}

Sobald ein Messwert in der Datenbank hinzugefügt wurde, wird die RuleEngine alarmiert,
welche wiederum eine Überprüfung des Messwertes durchführt. Ein gutes Beispiel hierfür ist die
Methode CheckNoiceSensorData, welche überprüft, ob der gemessene Wert über 300 db hat.

\begin{figure}[H]
    \centering
    \includegraphics[width=1.1\textwidth]{pics/CheckNoiceData.JPG}  
    \caption{Kontrollieren eines Messwertes von dem Noicesensor}
\end{figure}

Sobald die Methode CheckNoiceSensorData ausgeführt wurde und der gemessene Wert über 300 db liegt,
wird der Benutzer über Telegram verständigt. Aus diesem Grund wurde die Klasse MessageNotification, 
welche die Methode SendMessageByTelegram beinhaltet, geschrieben. Wie der Name SendMessageByTelegram
schon sagt, wird in dieser Methode eine Message mittels Telegram über einen Bot versendet.
Für diesen Vorgang wird der Token, welcher für den Zugriff der HTTP API zuständig ist, und die ChatId benötigt.
Diese sind in einem appsettings.json File gespeichert und werden über die Variable configuration eingelesen.

\begin{figure}[H]
    \centering
    \includegraphics[width=1.1\textwidth]{pics/MessageTelegramConfiguration.JPG}  
    \caption{Configuration Daten für das Senden der Benachrichtigung}
\end{figure}

\begin{figure}[H]
    \centering
    \includegraphics[width=0.8\textwidth]{pics/MessageTelegramUsage.JPG}  
    \caption{Verwendung von der Configuration}
\end{figure}

\begin{figure}[H]
    \centering
    \includegraphics[width=0.7\textwidth]{pics/MessagesÜberSendByTelegram.JPG}  
    \caption{Veranschaulichung von gesendeten Benachrichtigungen}
\end{figure}


\subsection{SnQPoolIot.AspMvc}

Für das Veranschaulichen von Sensoren und gemessenen SensorDaten wurde eine AspMvc-Anwendung programmiert.

Eine Ansicht über aktuelle Sensoren wird folgendermaßen angezeigt:

\begin{figure}[H]
    \centering
    \includegraphics[width=1\textwidth]{pics/SensorenInDerAspMvcAnwendung.JPG}  
    \caption{Veranschaulichung von den Sensoren in der AspMvc Anwendung}
\end{figure}