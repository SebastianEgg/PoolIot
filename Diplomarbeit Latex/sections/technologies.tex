\section{Raspberry Pi}
\setauthor{Sebastian Egger}



\section{MQTT}
\setauthor{Sebastian Egger}


\section{.Net}
\setauthor{Sebastian Egger}

Projekte\\
Das Backend setzt sich aus 12 Projekten zusammen.

\begin{itemize}
    \item CommoneBase
\end{itemize}

In CommoneBase befinden Klassen und Methoden, die wiederverwendbar sind um Code verdoppelung zu vermeiden.

\begin{itemize}
    \item CSharpCodeGenerator.Logic
\end{itemize}

CSharpCodeGenerator.Logic dient zum automatischen generieren von den Entitäten 
in SnQPoolIot.Logic, SnQPoolIot.WebApi und SnQPoolIot.AspMvc.
In SnQPoolIot.Contracts werden Entitäten als Interfaces angegeben.
Wenn danach die Solution gebuildet wird, werden für die Entitäten Klassen und die dazugehörigen Controller angelegt.
Es werden auch Beziehung und Keys zwischen den Entitäten generiert.


\begin{itemize}
    \item SnQPoolIot.Adapters
\end{itemize}

SnQPoolIot.Adapters bietet uns einen direkten Zugriff auf die Logic.
Der Zugriff auf die Logic kann dadurch entweder direkt erfolgen oder per Rest über die WebApi.

\begin{itemize}
    \item SnQPoolIot.WebApi
\end{itemize}

Der Zugriff auf die Messwerte wird durch Rest-Zugriffe in SnQPoolIot.WebApi providet.
Auf die Daten kann aber nur zugegriffen werden, wenn man sicher vorher mit einem Account einloggt.  

\begin{itemize}
    \item SnQPoolIot.Contracts
\end{itemize}

SnQPoolIot.Contracts beinhaltet alle notwendigen Schnittstellen und Enumerationen des Projektes.
Hier werden die Entitäten als Interfaces angelegt, 
welche von CSharpCodeGenerator.Logic als Klassen und Controller in den oben genanten Projekten generieren.

\begin{itemize}
    \item SnQPoolIot.Logic
\end{itemize}

SnQPoolIot.Logic ist das Kernstück des Projektes. Es beinhaltet 
