\section{Raspberry Pi}
\setauthor{Sebastian Egger}



\section{MQTT}
\setauthor{Sebastian Egger}


\section{.Net}
\setauthor{Sebastian Egger}

Projekte\\

Das Backend setzt sich aus 12 Projekten zusammen, welche in C\# .Net 5 geschrieben sind.

\begin{itemize}
    \item CommonBase
\end{itemize}

In CommonBase befinden Klassen und Methoden, die wiederverwendbar sind um Code verdoppelung zu vermeiden.

\begin{itemize}
    \item CSharpCodeGenerator.Logic
\end{itemize}

CSharpCodeGenerator.Logic dient zum automatischen generieren von den Entitäten 
in SnQPoolIot.Logic, SnQPoolIot.WebApi und SnQPoolIot.AspMvc.
In SnQPoolIot.Contracts werden Entitäten als Interfaces angegeben.
Wenn danach die Solution gebuildet wird, werden für die Entitäten Klassen und die dazugehörigen Controller angelegt.
Es werden auch Beziehung und Keys zwischen den Entitäten generiert.


\begin{itemize}
    \item SnQPoolIot.Adapters
\end{itemize}

SnQPoolIot.Adapters bietet uns einen direkten Zugriff auf die Logic.
Der Zugriff auf die Logic kann dadurch entweder direkt erfolgen oder per Rest über die WebApi.

\begin{itemize}
    \item SnQPoolIot.WebApi
\end{itemize}

Der Zugriff auf die Messwerte wird durch Rest-Zugriffe in SnQPoolIot.WebApi providet.
Auf die Daten kann aber nur zugegriffen werden, wenn man sicher vorher mit einem Account einloggt.  

\begin{itemize}
    \item SnQPoolIot.Contracts
\end{itemize}

SnQPoolIot.Contracts beinhaltet alle notwendigen Schnittstellen und Enumerationen des Projektes.
Hier werden die Entitäten als Interfaces angelegt, 
welche von CSharpCodeGenerator.Logic als Klassen und Controller in den oben genanten Projekten generieren.

\begin{itemize}
    \item SnQPoolIot.Logic
\end{itemize}

SnQPoolIot.Logic ist das Kernstück des Projektes. 
Durch die Logic können wir alle Daten von der Datenbank holen. 
Die Datenbank auf welche die Logik zugreift ist ein SqlServer und der Zugriff und das erzeugen der Datenbank wird mittels Entityframework

\begin{itemize}
    \item SnQPoolIot.Transfer
\end{itemize}

SnQPoolIot.Transfer verwaltet die Transferobjekte für den Datenaustausch zwischen den Layers.

\begin{itemize}
    \item SnQPoolIot.AspMvc
\end{itemize}

SnQPoolIot.AspMvc ist ein Ersatz für das Frontend.
Hier werden die Funktionen wie zum Beispiel einloggen eines Users oder anzeigen von Messwerten dargestellt.

\begin{itemize}
    \item SnQPoolIot.ConApp
\end{itemize}

In SnQPoolIot.ConApp werden User mit verschiedenen Rechten angelegt, die für die Authentifizierung benötigt werden.

.Net 5 vs .Net 6
Unser Backend wurde wie oben bereits erwähnt mit .Net 5 geschrieben, weil die Umstellung zu .Net 6 einiges mit sich bringt.
Der Hauptgrund warum wir uns für .Net 5 entschieden haben, ist die Unterscheidung der Nullability von Objekten und die dadurch ausgelösten Warnings.
In .Net 5 werden durch Objekte die Null sein könnten keine Warnings angezeigt und man ersparrt sich etliche Zeit, wenn man sich nicht durch die Warnings käpfen muss.
Jedoch im Vergleich zu .Net 5 hat .Net 6 eine bessere Leistung, denn bei einer Ausführung in der Cloud sinkt es die Computekosten, welches ich als riesigen Vorteil sehe.
Einen weiteren Nachteil finde ich ist die Lesbarkeit des Code.
Hier ein Beispiel zu diesem Thema:

[Foto von ConApp .Net 5] [Foto von ConApp in .Net 6]


